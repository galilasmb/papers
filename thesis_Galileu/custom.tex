
% ----------------------------------------------------------
% DADOS DO TRABALHO - CAPA e FOLHA DE ROSTO
% Configure os dados do trabalho aqui
% ----------------------------------------------------------
\titulo{\textbf{A Lightweight Technique for Detecting Semantic Conflicts with Static Analysis}}
\autor{Galileu Santos de Jesus}
\local{Recife}
\data{2026}
\areaconcentracao{\textbf{Area of Concentration}: Software Engineering and Programming Languages}
\orientador{\textbf{Advisor}: Paulo Henrique Monteiro Borba}
\coorientador{\textbf{Co-advisor}: Rodrigo Bonifácio de Almeida}

\instituicao{UNIVERSIDADE FEDERAL DE PERNAMBUCO \\ CENTRO DE INFORMÁTICA \\ PROGRAMA DE PÓS-GRADUAÇÃO EM CIÊNCIA DA COMPUTAÇÃO}
\departamento{Centro de Informática}
\programa{Pós-graduação em Ciência da Computação}
\emailprograma{gsj@cin.ufpe.br}
\siteprograma{http://cin.ufpe.br/\textasciitilde posgraduacao}

\tipotrabalho{Ph.D. Thesis}
% O preambulo deve conter o tipo do trabalho, o objetivo, 
% o nome da instituição e a área de concentração 
%\preambulo{Trabalho apresentado ao Programa de Pós-graduação em Ciência da Computação do Centro de Informática da Universidade Federal de Pernambuco, como requisito parcial para obtenção do grau de Mestre Profissional em Ciência da Computação.}

%\preambuloatadefesa{Dissertação apresentada ao Programa de Pós-Graduação Profissional em Ciência da Computação da Universidade Federal de Pernambuco, como requisito parcial para a obtenção do título de Mestre Profissional em 04 de setembro de 2020.}

\preambulo{A Ph.D. Thesis presented to the Center for Informatics of Federal University of Pernambuco in partial fulfillment of the requirements for the degree of Philosophy Doctor in Computer Science.}

\preambuloatadefesa{Doctoral Thesis submitted to the Graduate Program in Computer Science at the Federal University of Pernambuco, as a partial requirement for obtaining the degree of Doctor in Computer Science. Area of Concentration: Software Engineering and Programming Languages.}





